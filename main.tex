\documentclass{article}
\usepackage[utf8]{inputenc}
\usepackage[spanish]{babel}
\usepackage{listings}
\usepackage{graphicx}
\graphicspath{ {images/} }
\usepackage{cite}

\begin{document}

\begin{titlepage}
    \begin{center}
        \vspace*{1cm}
            
        \Huge
        \textbf{Calistenia}
            
        \vspace{0.5cm}
        \LARGE
        Subtítulo
            
        \vspace{1.5cm}
            
        \textbf{Alejandro Lopera Gutierrez}
            
        \vfill
            
        \vspace{0.8cm}
            
        \Large
        Despartamento de Ingeniería Electrónica y Telecomunicaciones\\
        Universidad de Antioquia\\
        Medellín\\
        Marzo de 2021
            
    \end{center}
\end{titlepage}

\tableofcontents
\newpage
\section{Sección introductoria}\label{intro}
Esta es la primera sección, podemos agregar algunos elementos adicionales y todo será escrito correctamente. Más aún, si una palabra es demasiado larga y tiene que ser truncada, babel tratará de truncarla correctamente dependiendo del idioma.

\section{Sección de contenido} \label{contenido}
Esta sección es para agregar toda la información correspondiente con código, citas, etc.
\subsection{Citación}
Vamos a citar por ejemplo un artículo de \textbf{Albert Einstein} \cite{einstein}.
También es posible citar libros \cite{dirac} o documentos en línea \cite{knuthwebsite}.\\\\
Revisar en la última sección el formato de las referencias en IEEE.

\subsection{Incluir código en el documento}
%
A continuación, se presenta el código \ref{codigo_ejemplo}, que nos permite incluir en el informe partes de programa que requieran una explicación adicional.
\begin{lstlisting}[language=C++, label=codigo_ejemplo]
// Programa desarrollado, compilado y ejecutado en https://www.onlinegdb.com
#include <iostream>

/*
 * Esto es un comentario de varias lineas
 */

// Comentario de una sola linea

#define N 10

using namespace std;

int main()
{
    
    for( int i = 0 ; i < N ; i++ ){
        
        if( !(i % 2) )
            cout << " El valor de i es -> " << i << endl;
    }
    
    return 0;
}

//Resultado programa

/*
El valor de i es -> 0
El valor de i es -> 2
El valor de i es -> 4
El valor de i es -> 6
El valor de i es -> 8
*/
\end{lstlisting}
En la sección \ref{imagenes}, se presentará como añadir ilustraciones al texto.

\section{Inclusión de imágenes} \label{imagenes}

En la Figura (\ref{fig:cpplogo}), se presenta el logo de C++ contenido en la carpeta images.

\begin{figure}[h]
\includegraphics[width=4cm]{cpplogo.png}
\centering
\caption{Logo de C++}
\label{fig:cpplogo}
\end{figure}

Las secciones (\ref{intro}), (\ref{contenido}) y (\ref{imagenes}) dependen del estilo del documento.


\section{Plantamiento del problema}\label{Plantamiento del problema}
Se debe forma un triangulo con 2 tarjetas encima de una base de papel y una de las reglas es que solo puedes usar una mano. Para poder lograr este objetivo debes de seguir las siguientes instrucciones 

\section{Instrucciones}\label{Instrucciones}
1. Tener una hoja de papel y dos tarjetas del mismo tamaño 

2. Juntar las tarjetas y que queden posicionadas una sobre otra

3. Cuando se cumpla el paso anterior deberes de poner la hoja encima de las tarjetas

4. Después de haber cumplido el paso anterior levantas la hoja y la pones a un lado de la mesa

5. Si has seguido los pasos acorde a lo establecido ya deberás de coger y poner las tarjetas encima de la hoja de papel

6. Cuando las tarjetas estén posicionadas encima de la hoja de papel deben intentar poder armar un triangulo 

7. Con los dedos te vas ayudar y así lograras cumplir el objetivo

8. Usa el pulgar, el indice y el dedo del medio para agilizar y tener un mejor manejo de las tarjetas 


\section{Conclusiones}\label{Conclusiones}
Gracias a  este trabajo aprendi a como dar mejores ordenes y tener un mayor liderazgo hacia las personas aunque como se puede ver en el video vario de ellos no siguieron a la perfeccion las ordenes que les entregue pero aun asi lograron la meta de poder armar el triangulo con las dos tarjetas

\bibliographystyle{IEEEtran}
\bibliography{references}

\end{document}
